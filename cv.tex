% xelatex
\documentclass[a4paper,
		%twocolumn,
		10pt]{article}
\usepackage[utf8]{inputenc}
\usepackage[norsk]{babel}
\usepackage{metalogo}
\usepackage{xifthen}
\usepackage[colorlinks=true,urlcolor=Blue]{hyperref}
\usepackage{graphicx}
\usepackage{fontspec}
\usepackage[T1]{fontenc}
\usepackage[dvipsnames]{xcolor}
\usepackage{titlesec}
\usepackage[margin=1in]{geometry}
\usepackage{titling}
\newfontfamily\cfont{Noto Sans CJK SC}
\usepackage{libertine}

% Macro to allow image links in XeLaTeX
\ifxetex
  \usepackage{letltxmacro}
  \setlength{\XeTeXLinkMargin}{1pt}
  \LetLtxMacro\SavedIncludeGraphics\includegraphics
  \def\includegraphics#1#{% #1 catches optional stuff (star/opt. arg.)
    \IncludeGraphicsAux{#1}%
  }%
  \newcommand*{\IncludeGraphicsAux}[2]{%
    \XeTeXLinkBox{%
      \SavedIncludeGraphics#1{#2}%
    }%
  }%
\fi
%%%%%%%

% Bold contents of a link
\let\oldhref\href
\renewcommand{\href}[3][blue]{\oldhref{#2}{\color{#1}{#3}}}

% Your name goes here:
\author{Gunnar K. Halvorsen}

% Update date set to last compile:
\date{\today}

% Custom title command.
\renewcommand{\maketitle}{
	\hspace{.25\textwidth}
	\begin{minipage}[t]{.5\textwidth}
\par{\centering{\Huge  \bfseries{\theauthor}}\par}
	\end{minipage}
	\begin{minipage}[t]{.25\textwidth}
{\footnotesize\hfill{}\color{gray}
\hfill{}Last ned dokumentet:

\hfill{}\href[gray]{https://ghalv.no/cv.pdf}{https://ghalv.no/cv.pdf}

\hfill{}(Revidert \thedate.)
}
	\end{minipage}
}



% Setting the font I want:
\renewcommand{\familydefault}{\sfdefault}
\usepackage{sqrcaps}

% Making the \entry command
\newcommand{\entry}[4]{
\ifthenelse{\isempty{#3}}
{\slimentry{#1}{#2}}{

\begin{minipage}[t]{.15\linewidth}
\hfill \textsc{#1}
\end{minipage}
\hfill\vline\hfill
\begin{minipage}[t]{.80\linewidth}
{\bf#2}\\\textit{#3} \footnotesize{#4}
\end{minipage}\\
\vspace{.2cm}
}}

\newcommand{\slimentry}[2]{

\begin{minipage}[t]{.15\linewidth}
\hfill \textsc{#1}
\end{minipage}
\hfill\vline\hfill
\begin{minipage}[t]{.80\linewidth}
#2
\end{minipage}\\
\vspace{.25cm}
}% end \entry command definition

% Some macros because I'm lazy:
\newcommand{\uis}{Universitetet i Stavanger}
\newcommand{\imf}{institutt for matematikk og fysikk (IMF) }

\let\lineheight\baselineskip

% Link images
\newcommand{\pdf}{\includegraphics[height=.85em]{cv/pdf.png}}
\newcommand{\yt}{\includegraphics[height=.85em]{cv/yt.png}}
\newcommand{\gh}{\includegraphics[height=.85em]{cv/gh.png}}
\newcommand{\lin}{\includegraphics[height=.85em]{cv/lin.png}}
\newcommand{\www}{\includegraphics[height=.85em]{cv/www.png}}
\newcommand{\email}{\includegraphics[height=.85em]{cv/email.png}}

% Custom section spacing and formatting
\titleformat{\part}{\Huge\scshape\filcenter}{}{1em}{}
\titleformat{\section}{\Large\bf\raggedright}{}{1em}{}[{\titlerule[2pt]}]
\titlespacing{\section}{0pt}{3pt}{7pt}
\titleformat{\subsection}{\large\bfseries\centering}{}{0em}{\underline}%[\rule{3cm}{.2pt}]
\titlespacing{\subsection}{0pt}{7pt}{7pt}

% No indentation
\setlength{\parindent}{0in}

\begin{document}

\maketitle

\section{Bio}

\begin{minipage}[t]{.5\linewidth}
\begin{tabular}{rp{.75\linewidth}}
	\baselineskip=20pt
	\email{} :     & \href{mailto:gunnar@ghalv.no}{gunnar@ghalv.no}\\
	\www{} : &\href{https://www.ghalv.no}{https://ghalv.no}
\end{tabular}
\end{minipage}
\begin{minipage}[t]{.5\linewidth}
\begin{tabular}{rl}
 	\lin{} : & \href{https://www.linkedin.com/in/ghalv}{linkedin.com/in/ghalv}\\
        \gh{} : & \href{http://github.com/ghalv}{github.com/ghalv}
\end{tabular}
\end{minipage}

\begin{itemize}
\item bakgrunn innen matematikk/statistikk, men ser i grunnen mer på meg selv som en datatype. Hovedtyngden min er innen linux-systemer, som jeg bruker daglig både som programmeringsmiljø, men også til personlig epost- og webserver (VPS). Jeg behersker shell, R, vim og git på et forholdsvis høyt nivå, samt en rekke andre teknologier på et mer familiært nivå. E.g. python, LaTeX, docker, nginx, html, css.

\item Etter samtalen så jeg over Microsoft-produktene vi snakket om (i.e. Power Platform, DevOps, Azure), og har satt meg litt inn i pensum til AZ-900. Det slår meg som overkommelig materiale, og en retning jeg kan være interessert i å følge videre. DevOps har jeg som nevnt vært innom, men da i begrenset forstand.
 
\item Som person er jeg sosialt anlagt, og samarbeider godt med andre. Referanse kan oppgis og CV ligger vedlagt.
\end{itemize}

\section{Erfaring}

\entry{2022--}
	{Forskningsassistent}
	{\uis}
	{SPEDAIMS er et nasjonalt forskningsprosjekt som søker å kartlegge lese- og tallforståelse. Min rolle er å observere og utføre tester for å kartlegge kandidatene. Har bistått med utvikling av app (kotlin), og domene-forwarding.}

\entry{2018--2022}
	{Instituttrådsmedlem ved \imf}
	{\uis}
	{Satt som valgt studentrepresentant i flere perioder. Så til at studentenes perspektiv ble ihensyntatt.}

 \entry{2019}
        {Ambassadør ved \imf}
        {\uis}
        {Holdt foredrag om utdanning i matematikk og fysikk på Sandnes vgs, og var representant for studiet i flere år under åpen dag.}

\entry{2018}
	{Ansettelseskomité for instituttleder ved \imf}
	{\uis}
	{Var med på utlysning, intervju og ansettelse av instituttleder.}

\entry{2018--2022}
	{Studentparlamentsmedlem, \uis}
	{Studentorganisasjonen ved UiS, StOr}
	{Leder for initiativet 'Kontrær Liste'. Fikk utmerkelse fra StOr og hadde innlegg på trykk i Stavanger Aftenblad.}

 \entry{2017--2022}
        {Nestleder for Theta, linjeforening for matematikk- og fysikkstudenter ved UiS}
	{Studentorganisasjonen ved UiS, StOr}
 	{Nestleder for linjeforeningen i flere perioder. Utviklet nettsiden, og var med på utarbeidelse av kompendium i hhv. kalkulus og Newtonisk mekanikk.}

\entry{2018}
        {Jurymedlem for Lyses forskningspris 2017}
	{Lyse Energi}
 	{Satt som representant under tildelingen for 2017. Vurderte blablabla}

\entry{2014--2015}
        {Butikkmedarbeider}
        {Elektronikkforhandler}

\entry{2012--2013}
        {Lærervikar Soma skole}
        {Sandnes Kommune}
        {Jobbet ved Soma og Stangeland skole. Hadde bl.a. med en 6. klasse på Knaben leirskole.}

\entry{2011}
        {MIF Comcen-operatør}
        {INI Operations}
        {Representant for tillitsmannsordningen, TMO}

\entry{2011}
        {Konstabel} 
        {Kgl. Norske Marine}
        {Befalsansvar for menige ved avd. Høyeste vurdering på tjenesteuttalelse.}

\entry{2007--2008}
        {Butikkmedarbeider}
        {Coop Økonom BA}

\section{Utdanning}

\entry{2021--}
        {Masterprogram i matematikk og fysikk}
        {\uis}
        {Spesialisering i statistikk.}

\entry{2015--2018}
	{Matematikk og fysikk}
	{\uis}
        {Initiativtaker for Linux User Group og Latinforbundet. Førstnevnte lagde skript som hentet relevante busstider via Kolumbus' API og fikk støtte fra Kunnskapsdep. i sak om digital eksamen (2017).}

\entry{2013--2014}{Spesiell studiekompetanse}
	{Stavanger Katedralskole}
 	{Matematikk, fysikk og kjemi.}

\entry{2008-2010}
        {Generell studiekompetanse}
	{Bryne videregående skole}
	{Medlem av elevrådet.}

\section{Språk}

\entry{Uttalte}
	{Norsk, engelsk, grunnleggende latin.}
	{}
	{}

\entry{Programmering}
	{R (tidyverse), bash/shell, Python, overfladisk kjennskap til C.}
	{}
	{}

\entry{Markup}
	{{\LaTeX}/{\XeTeX}, RMarkdown, HTML og CSS.}
        {}

\section{Utvalgte prosjekter}

\subsection{Typisk arbeidsflyt}

I use a \textbf{vim}-based setup in a tiling window manager (\textbf{i3-gaps}). I compile documents using \textbf{R Markdown} or \textbf{\LaTeX}, and \textbf{biber} for references. I prefer to do multimedia manipulation in the terminal with tools like \textbf{imagemagick} and \textbf{ffmpeg} for extensibility's sake.
I've run Microsoft, MacOS and GNU/Linux systems (both Debian and Arch-based varieties, as well as Void Linux).

\subsection{Programmer jeg er fotrolig med}

API/graphql, -- tmux, ssh, RStudio, Blender (mostly for video-editing), Praat, Audacity, E-Prime, GIMP, pandoc, Jupyter. I've managed websites manually via ssh and vim using HTML/CSS/PHP and with tools such as Github Pages (Jekyll) WordPress via either cpanel or wp-cli. Some experience with myBB.


\section{Interesser}

Classical languages, human evolution and prehistory, free (i.e. libre) software, survivalism, cybernetics, medieval thought, Rhaeto-Romance poetry. 

\section{Referanse}

\begin{itemize}
\item Email me \href{mailto:gunnar@ghalv.no}{\email} with your interests in me and I'll refer you to someone who can vouch for me or defame me, depending on what you want.
\end{itemize}


\end{document}