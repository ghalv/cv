% xelatex
\documentclass[a4paper,
		%twocolumn,
		10pt]{article}
\usepackage[utf8]{inputenc}
\usepackage[norsk]{babel}
\usepackage{metalogo}
\usepackage{xifthen}
\usepackage[colorlinks=true,urlcolor=Blue]{hyperref}
\usepackage{graphicx}
\usepackage{fontspec}
\usepackage[T1]{fontenc}
\usepackage[dvipsnames]{xcolor}
\usepackage{titlesec}
\usepackage[margin=1in]{geometry}
\usepackage{titling}
\newfontfamily\cfont{Noto Sans CJK SC}
\usepackage{libertine}

% Macro to allow image links in XeLaTeX
\ifxetex
  \usepackage{letltxmacro}
  \setlength{\XeTeXLinkMargin}{1pt}
  \LetLtxMacro\SavedIncludeGraphics\includegraphics
  \def\includegraphics#1#{% #1 catches optional stuff (star/opt. arg.)
    \IncludeGraphicsAux{#1}%
  }%
  \newcommand*{\IncludeGraphicsAux}[2]{%
    \XeTeXLinkBox{%
      \SavedIncludeGraphics#1{#2}%
    }%
  }%
\fi
%%%%%%%

% Bold contents of a link
\let\oldhref\href
\renewcommand{\href}[3][blue]{\oldhref{#2}{\color{#1}{#3}}}

% Your name goes here:
\author{Gunnar K. Halvorsen}

% Update date set to last compile:
\date{\today}

% Custom title command.
\renewcommand{\maketitle}{
	\hspace{.25\textwidth}
	\begin{minipage}[t]{.5\textwidth}
\par{\centering{\Huge  \bfseries{\theauthor}}\par}
	\end{minipage}
	\begin{minipage}[t]{.25\textwidth}
{\footnotesize\hfill{}\color{gray}
\hfill{}Last ned dokumentet:

\hfill{}\href[gray]{https://ghalv.no/cv.pdf}{https://ghalv.no/cv.pdf}

\hfill{}(Revidert \thedate.)
}
	\end{minipage}
}



% Setting the font I want:
\renewcommand{\familydefault}{\sfdefault}
\usepackage{sqrcaps}

% Making the \entry command
\newcommand{\entry}[4]{
\ifthenelse{\isempty{#3}}
{\slimentry{#1}{#2}}{

\begin{minipage}[t]{.15\linewidth}
\hfill \textsc{#1}
\end{minipage}
\hfill\vline\hfill
\begin{minipage}[t]{.80\linewidth}
{\bf#2}\\\textit{#3} \footnotesize{#4}
\end{minipage}\\
\vspace{.2cm}
}}

\newcommand{\slimentry}[2]{

\begin{minipage}[t]{.15\linewidth}
\hfill \textsc{#1}
\end{minipage}
\hfill\vline\hfill
\begin{minipage}[t]{.80\linewidth}
#2
\end{minipage}\\
\vspace{.25cm}
}% end \entry command definition

% Some macros because I'm lazy:
\newcommand{\uis}{Universitetet i Stavanger}
\newcommand{\imf}{institutt for matematikk og fysikk (IMF) }

\let\lineheight\baselineskip

% Link images
\newcommand{\pdf}{\includegraphics[height=.85em]{cv/pdf.png}}
\newcommand{\yt}{\includegraphics[height=.85em]{cv/yt.png}}
\newcommand{\gh}{\includegraphics[height=.85em]{cv/gh.png}}
\newcommand{\lin}{\includegraphics[height=.85em]{cv/lin.png}}
\newcommand{\www}{\includegraphics[height=.85em]{cv/www.png}}
\newcommand{\email}{\includegraphics[height=.85em]{cv/email.png}}

% Custom section spacing and formatting
\titleformat{\part}{\Huge\scshape\filcenter}{}{1em}{}
\titleformat{\section}{\Large\bf\raggedright}{}{1em}{}[{\titlerule[2pt]}]
\titlespacing{\section}{0pt}{3pt}{7pt}
\titleformat{\subsection}{\large\bfseries\centering}{}{0em}{\underline}%[\rule{3cm}{.2pt}]
\titlespacing{\subsection}{0pt}{7pt}{7pt}

% No indentation
\setlength{\parindent}{0in}

\begin{document}

\maketitle

\section{Bio}

\begin{minipage}[t]{.5\linewidth}
\begin{tabular}{rp{.75\linewidth}}
	\baselineskip=20pt
	\email{} :     & \href{mailto:gunnar@ghalv.no}{gunnar@ghalv.no}\\
	\www{} : & \href{https://www.ghalv.no}{https://ghalv.no}
\end{tabular}
\end{minipage}
\begin{minipage}[t]{.5\linewidth}
\begin{tabular}{rl}
 	\lin{} : & \href{https://www.linkedin.com/in/ghalv}{linkedin.com/in/ghalv}\\
        \gh{} : & \href{http://github.com/ghalv}{github.com/ghalv}
\end{tabular}
\end{minipage}

\begin{itemize}
	\item Bakgrunn innen matematikk/statistikk, men jobber som full stack-utvikler, og har erfaring med app-utvikling og web-programmering som konsulent og ansatt.

	\item Inngående kjennskap til Python, R, git, shell og linux-systemer (Arch, Debian). Har også jobbet med React Native, Firebase, node.js, filament/laravel, php, docker, GraphQL, ansible og relasjonsdatabaser (SQL).

	\item Er glad i å bistå med løsninger på tekniske problemer, ofte i form av kode. Sosialt anlagt som person, og samarbeider godt med andre.

\end{itemize}

\section{Erfaring}

\entry{2024--}
	{Programmerer}
	{Bambusa}
	{Backend-utvikler for web-basert dashboard skrevet i Filament/Laravel/LiveWire.}

\entry{2023--}
	{Dataingeniør}
	{Cerex}
	{Jobber med apputvikling i react native og computer vision.}

\entry{2023}
	{Lærer, deltid}
	{Sandnes Læringssenter}
	{Bidro i undervisning av Ukrainske flyktninger i Sandnes Kommune.}

\entry{2022--2023}
	{Forskningsassistent}
	{\uis}
	{SPEDAIMS er et nasjonalt forskningsprosjekt som søker å kartlegge lese- og tallforståelse. Min rolle er å observere og utføre tester for å kartlegge kandidatene. Laget software (python) som ble utgangspunkt for android app (kotlin), samt div. teknisk arbeid som domene-forwarding.}

\entry{2018--2022}
	{Instituttrådsmedlem ved \imf}
	{\uis}
	{Satt som valgt studentrepresentant i flere perioder. Så til at studentenes perspektiv ble ihensyntatt.}

 \entry{2019}
        {Ambassadør ved \imf}
        {\uis}
	{Holdt foredrag om utdanning i matematikk og fysikk på Sandnes vgs, og var representant for studiet i flere år under åpen dag. Lagde flyveblader.}

\entry{2018}
        {Jurymedlem, Lyses forskningspris 2017}
	{Lyse Energi}
 	{Vurderte syv nominerte kandidatur for tildelingen 2017. Prisen ble tildelt prof. Eva Johansson.}

\entry{2017}
	{Ansettelseskomité for instituttleder ved \imf}
	{\uis}
	{Satt i kommité for stillingsutlysning, intervju og ansettelse av instituttleder.}

\entry{2018--2022}
	{Studentparlamentsmedlem, \uis}
	{Studentorganisasjonen ved UiS, StOr}
	{Leder for initiativet 'Kontrær Liste'. Fikk utmerkelse av StOr og hadde innlegg på trykk i Stavanger Aftenblad.}

 \entry{2017--2022}
        {Nestleder for Theta, linjeforening for matematikk- og fysikkstudenter ved UiS}
	{Studentorganisasjonen ved UiS, StOr}
 	{Nestleder i flere perioder. Utviklet nettsiden som ble hostet på TEKNATs unix-nett, og var med på utarbeidelse av kompendium i hhv. kalkulus og Newtonisk mekanikk.}

\entry{2014--2015}
        {Butikkmedarbeider}
        {Elektronikkforhandler}

\entry{2012--2013}
        {Lærer}
        {Soma Skole}
        {Vikariat i Sandnes Kommune. Hadde bl.a. en 7. klasse med til Knaben leirskole.}

\entry{2011}
        {MIF Comcen-operatør, INI Operations}
        {Kgl. Norske Marine}
        {Konstabel med befalsansvar for meninge og representant for tillitsmannsordningen, TMO. Høyeste vurdering på tjenesteuttalelse.}

\entry{2007--2008}
        {Butikkmedarbeider}
        {Coop Økonom BA}

\section{Utdanning}

\entry{2015--2018}
	{Matematikk og fysikk}
	{\uis}
        {Initiativtaker for Linux User Group og Latinforbundet. Førstnevnte lagde skript som hentet relevante busstider via Kolumbus' API og fikk medhold fra Kunnskapsdep. i sak om digital eksamen (2017).}

\entry{2013--2014}{Spesiell studiekompetanse}
	{Stavanger Katedralskole}
 	{Matematikk, fysikk og kjemi.}

\entry{2008-2010}
        {Generell studiekompetanse}
	{Bryne videregående skole}
	{Medlem av elevrådet.}

\section{Sertifikater}

\entry{2023}
	{Unsupervised Learning, Recommenders, Reinforcement Learning}
	{Stanford online / DeepLearning.AI}
	{ID: \href{https://www.coursera.org/account/accomplishments/certificate/RJTW6ZD7FUWG}{RJTW6ZD7FUWG}}

\entry{2023}
	{Advanced Learning Algorithms}
	{Stanford online / DeepLearning.AI}
	{ID: \href{https://www.coursera.org/account/accomplishments/certificate/CYZMDL27UEGN}{CYZMDL27UEGN}}

\entry{2023}
	{Supervised Machine Learning: Regression and Classification}
	{Stanford online / DeepLearning.AI}
	{ID: \href{https://www.coursera.org/account/accomplishments/certificate/YXVDZ42BTVC7}{YXVDZ42BTVC7}}

\entry{2011}
	{Kryptokvalifisering grad 2}
	{Luftforsvarets skolesenter, Kjevik}
        {ID: 1381201}

\section{Språk}

\entry{Prog.}
	{Python, R, bash/shell, SQL, React Native/JavaScript, php, C.}
	{}
	{}

\entry{Markup}
	{{\LaTeX}/{\XeTeX}, RMarkdown, HTML og CSS.}
        {}

\entry{Uttalte}
	{Norsk, engelsk, grunnleggende latin.}
	{}
	{}

\section{Interesser}

Interesserer meg for friluftsliv, data, og historie. Leser en del, og går mye tur i skog og mark.

\section{Referanse}

\begin{itemize}
\item Referanse kan oppgis på henvendelse til \href{mailto:gunnar@ghalv.no}{gunnar@ghalv.no}.
\end{itemize}


\end{document}
